\documentclass{aastex631}

\shorttitle{Classification and characterization of  Changing-Look Quasars}

\begin{document}
\thispagestyle{empty}  % to remove the page number  for that single page ... 
%\title{}

\author{Krzysztof Suberlak}

Quasars are galaxies that harbor an actively accreting supermassive black hole. The majority of the optical light is the continuum emission from the hot accretion disk. This dynamic process results in light curves exhibiting a characteristic stochastic variability over time-scale of months-years at the root-mean-squared 0.2 mag level. An often employed mathematical description of  this behavior is the Damped Random Walk (DRW) model. However, the DRW model fails to account for sudden, large ($>0.5$ mag), long-term brightness changes observed for some objects. Such transitions have been associated with the (dis-)appearance of the broad emission line flux and changes in the continuum level. Similar behavior has been observed in less luminous, more nearby Changing-Look Active Galactic Nuclei (CLAGN). It was found that repeat spectroscopy, allowing an estimate of profile shape for major emission lines, is the key to understanding the mechanism driving the state change \citep{macleod2019}. Within the past few years several studies have compiled the available data, and reported that the Changing-Look QSO (CLQSO) phenomenon could pertain to as much as  15\% of all quasars, representing the tail end of the distribution described by the AGN flicker scenario. Recently, \citealt{suberlak2021} identified 40 CLQSO candidates by comparing the DRW timescales derived from shorter SDSS-only light curves to those using an extended SDSS-PS1 baseline. A cross-match against recent CLQSO studies showed that only one of these objects has been followed-up.  In comparison to the earlier spectroscopic epochs, it displayed dramatic changes in the strength of the Balmer line. Additional photometry from the Zwicky Transient Facility (ZTF), available for a subset of 12 candidates, revealed that some objects may be undergoing a turn-on event, while others showed signatures of periodicity, expected from a binary supermassive black hole system. An observing program of CLQSO follow-up spectroscopy (PI Suberlak) has been awarded 13 half-nights  on the Apache Point Observatory 3.5m telescope for the first two quarters of 2023. Since spectroscopy is very time-consuming (faint quasars require ~1hr exposure), a robust classification scheme is needed to find new candidates at the highest level of confidence, and allow performing an ensemble analysis. 
%This research uses the combined datasets from multiple surveys to find CLQSO candidates, and spectroscopy for their detailed follow-up.

Combining quasar light curves to detect CLQSOs is inherently a data science task. It involves data acquisition and  data cleaning - light curves from different surveys are often obtained in different photometric filters, and require processing before they can be analyzed as a single time-series. Data from different surveys needs to be either transformed via color terms into a common photometric system, or treated as a separate semi-simultaneous parallel time series for each object. Each survey is unique and requires a different approach to access the time series data for the same object (such as SQL or TAP  queries to online servers). The combined data is fit with a model using Gaussian Process, to provide a mathematical  description of light curve morphology. The model parameters can be linked to physical properties of each object, eg. black hole mass, or absolute luminosity.  Further spectroscopic characterization  also involves employing diverse datasets, such as raw science exposures, image calibrations, wavelength standards, to arrive at a calibrated spectrum. All this involves tools and techniques that are pertinent to generic data analysis tasks: python programming language (including astropy, scipy, pandas, matplotlib, and other packages), SQL queries, live visualization with jupyter notebooks. As the construction of the Rubin Observatory is in its final stages, this is a special moment to combine the data science tools and techniques to leverage the potential available in the upcoming new era of time domain astronomy. The DP0.2 data preview - a precursor dataset to prepare for the Legacy Survey of Space and Time (LSST) - allows testing data queries (for instance for quasar classification), before the live data arrives (with first light planned for late 2024). Approaches tested on a small number of objects need to be streamlined to work efficiently with a much larger LSST data stream, to quickly classify and follow-up CLQSOs. The UW Data Science Postdoctoral Fellowship creates unique opportunities for collaboration and sharing ideas with other data science-minded researchers; it would help me incorporate innovative tools and techniques that can help expand the search for CLQSO to the LSST scale. Often solutions from one research application can help in a very different context. For instance, as a Data Science for Social Good fellow, I employed hierarchical clustering  (used in astronomy), to evaluate avenues for improvement for programs that help mitigate homelessness. I believe that my experience, including that of commissioning the Active Optics System for Rubin Observatory, could likewise contribute to the perspectives of other postdocs in the program, and by sharing solutions aid their research. 

\bibliography{references}{}
\bibliographystyle{aasjournal}


\end{document}